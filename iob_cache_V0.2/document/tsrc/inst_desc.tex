% SPDX-FileCopyrightText: 2024 IObundle
%
% SPDX-License-Identifier: MIT

The figure shows a three-level memory hierarchy comprising L1 and L2 caches and
a memory module. The Host drives the L1 cache through its front-end NPI
interface (the user is free to develop other interfaces).  The L1 and L2 caches
are connected using another NPI interface since IOb-Cache's front-end interface
only supports this.

The {\tt wtb\_empty\_in} and {\tt wtb\_empty\_out} signals form a chain from the
L1's front-end to the L2's back-end. As explained in the description of these
signals, this chain ensures that the user's core knows that all write-through
buffers across the cache hierarchy are empty. Note that the L1's {\tt
  wtb\_empty\_out} signal is floating because the Host uses the cache controller
to query the write-through buffer status. The L2's {\tt wtb\_empty\_in} is tied
high as L2 is the last cache in the hierarchy, and there are no more
write-through buffers to its right-hand side.

The {\tt invalidate\_in} and {\tt invalidate\_out} signals form another chain that
ensures that the data in the whole cache hierarchy is invalidated, as explained
in these signal's descriptions. Note that the L1's {\tt invalidate\_in} signal is
tight to low as L1 is invalidated via the cache controller by writing to the
respective address. The L2's {\tt invalidate\_out} signal is floating because L2
is the last cache in the hierarchy, and there are no more caches to invalidate.

Finally, L2 is connected to a memory module, and one can choose between NPI or
AXI4 interfaces. In practice, most memory modules have a standard interface such
as AXI4, which is themost common choice, although one may choose NPI in less
usual simulation or FPGA prototyping scenarios.

